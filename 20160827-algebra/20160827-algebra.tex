\documentclass{article}

\usepackage{fontspec}
\usepackage{amsmath}
\usepackage{amssymb}

\setmainfont{AR PL UKai CN}

\XeTeXlinebreaklocale "zh"
%\XeTeXlinebreakskip = 0pt plus 1pt

\DeclareMathOperator{\rank}{Rank}

\title{2016大数竞赛模拟试题(代数与几何)}
\author{}
\date{2016-08-27}

\begin{document}

\maketitle

\begin{enumerate}
  \item
    求以原点为顶点,
    与$z = 3 x^2 + 3 y^2 + 1$相切的锥面方程.
  \item
    设$n$阶方阵$B(t)$与$n$维列向量$b(t)$的每个元素都为关于$t$的实系数多项式.
    记$B(t)$的行列式为$d(t)$,
    $B(t)$的第$i$列被$b(t)$替换后的行列式为$d_i(t)$.
    已知$d(t)$有实根$t_0$,
    且关于$x$的方程组$B(t_0) x = b(t_0)$是相容的.
    试证明$d(t),d_1(t),\dotsc,d_n(t)$的最大公因式次数至少为1.
  \item
    求行列式
    \begin{equation}
      \left|
      \begin{array}{cccc}
        \sin \theta_1 & \sin 2 \theta_1 & \cdots & \sin n \theta_1 \\
        \sin \theta_2 & \sin^2 \theta_2 & \cdots & \sin^n \theta_2 \\
        \vdots & \vdots & & \vdots \\
        \sin \theta_n & \sin^2 \theta_n & \cdots & \sin^n \theta_n
      \end{array}
      \right|
    \end{equation}
  \item
    设$V$为闭区间$[0,1]$上的所有实值函数在通常加法与数乘意义下构成的实向量空间.
    试证明$V$中元素$f_1,\dotsc,f_n$线性无关当且仅当存在$a_1,\dotsc,a_n \in [0,1]$使得行列式$\det \left[ f_i(a_j) \right] \neq 0$.
  \item
    证明对于任意(可相乘)矩阵, 必有
    \begin{equation}
      \rank (AB) + \rank (BC) \leq \rank (ABC) + \rank (B)
    \end{equation}
  \item
    设$V$是$n$维实线性空间, 试证明:
    \begin{enumerate}
      \item
        $V$的任意有限多个真子空间的并不为$V$.
      \item
        对于任意的$m>n$,
        可取得$V$中$m$个向量,
        使得其中任意$n$个均构成$V$的一组基.
      \item
        利用(a)的结果证明(b).
      \item
        利用(b)的结果证明(a).
    \end{enumerate}
\end{enumerate}

\end{document}
